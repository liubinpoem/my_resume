%-------------------------------------------------------------------------------
%	SECTION TITLE
%-------------------------------------------------------------------------------
\vspace{-5pt}
\cvsection{项目经历}
%-------------------------------------------------------------------------------
%	CONTENT
%-------------------------------------------------------------------------------
\begin{cventries}

%---------------------------------------------------------
\cventry
    { } % Affiliation/role目标:研究朴素贝叶斯算法对短文本分类的准确度及应用
    {汇兑账号挖掘打击} % Organization/group
    {02/2019 - 至今} %Date(s)
    {深圳/腾讯} % Location
    {
    \vspace{-8pt}
      \begin{cvitems} % Description(s) of experience/contributions/knowledge
      	\item{\textbf{项目目的:}国家对反洗钱管控越来越严格,需要从账号中找出经营汇兑业务的账号,净化支付环境}
	\item{进行中}
      \end{cvitems}
    }


  \cventry
    { } % Affiliation/role目标:研究朴素贝叶斯算法对短文本分类的准确度及应用
    { 基于出入账文本的风险账号挖掘项目} % Organization/group
    {12/2018 - 至今} %Date(s)
    {深圳/腾讯} % Location
    {
    \vspace{-8pt}
      \begin{cvitems} % Description(s) of experience/contributions/knowledge
      	\item{\textbf{项目目的:}利用出入帐文本信息,挖掘赌博、诈骗、色情等风险账号,净化支付环境}
        \item {从真实数据库中,选择赌博和微商账号构造数据集;在GPU Docker中安装TensorFlow以及Keras}
        \item {利用全量账号的转账文本挖掘新词,结合结巴分词的词典、funNLP以及搜狗开源词库,构造自定义词典;利用结巴分词处理数据}
        \item{利用Keras开发TextCNN模型,Gensim库得到词向量和文本向量,对比词向量和TFIDF方法作为向量化方法,和TextCNN模型、Xgboost模型、BERT等几个分类模型}
        \item{对比各个模型后得到最佳模型,每周挖掘出高精准的1w的赌博账号,然后写入账号系统进行打标、拦截和打击,以及高风险赌博用户进行限制,净化支付环境}
        \item{相关技能:Python、TensorFlow、Docker、GPU、深度学习、NLP、BERT}
      \end{cvitems}
    }
    
    \cventry
    {} % Affiliation/role目标:研究外卖美食畅销的主要影响因素
    {大规模无监督算法异常发现项目} % Organization/group
    {06/2018-11/2018} % Date(s)
    {腾讯/深圳} % Location
    {
    \vspace{-8pt}
      \begin{cvitems} % Description(s) of experience/contributions/knowledge
        \item {\textbf{项目目的:}有监督的方法需要大量的标注数据,标注数据成本高且难以做到非常准确,无监督方法可以避免这些弊端}
        \item {利用PrefixSpan算法进行特征组合,利用KL散度进行特征选择}
        \item {利用Spark实现论文《Outlier Detection in Complex Categorical Data by Modeling the Feature Value Couplings》中属性共现算法,衡量账号特征取值的异常度和整体账号的异常度}
        \item {实现Weighted MinHash算法,对计算得到异常度的账号进行聚类划分,然后计算每个聚类的异常度,把账号划分到异常度最高的聚类中去,异常度挖掘算法确实有效}
        \item {相关技能:Spark、Scala、Hive、HiveSQL、Hadoop}
      \end{cvitems}
    }
    
  \cventry
    { } % Affiliation/role目标:研究朴素贝叶斯算法对短文本分类的准确度及应用
    { PipeML自动化机器学习平台项目} % Organization/group
    {04/2018-08/2018} %Date(s)
    {腾讯/深圳} % Location
    {
    \vspace{-8pt}
      \begin{cvitems} % Description(s) of experience/contributions/knowledge
      	\item{\textbf{项目目的:}让团队成员无脑自动化建模,并得到机器学习模型}
        \item {利用Spring+MyBatis框架作为Web框架,jQuery+eCharts+BootStrap作为前端显示组件,MySQL作为后台数据库存储数据}
        \item {继承Python Sklearn库的类,开发自定义的模型,前台接收到数据请求后,调起Python算法脚本,进行自动化预处理、特征选择等操作}
        \item{在前台输出每个用户该次训练的模型指标、特征重要性排序、IV值柱状图,选择最合适的模型和最重要的特征}
        \item{利用PMML文件的方式和pickle文件的方式保存用户的模型,得到可用的模型结果}
        \item{相关技能:Python、Java、Sklearn、Spring、MyBatis、JavaScript、BootStrap、MySQL}
      \end{cvitems}
    }
    
    \cventry
    { } % Affiliation/role目标:研究朴素贝叶斯算法对短文本分类的准确度及应用
    {风险团伙可视化项目} % Organization/group
    {06/2017-09/2017} %Date(s)
    {腾讯/深圳} % Location
    {
    \vspace{-8pt}
      \begin{cvitems} % Description(s) of experience/contributions/knowledge
      	\item{\textbf{项目目的:}将利用SparkX和Pregel框架进行图算法挖掘出来的团伙进行可视化展示,方便审核定性}
        \item {以Django作为Web框架,进行数据展示;MySQL作为数据存储库;采用D3JS技术展示团伙}
        \item {在团伙展示中,以丰富的数据信息使团伙尽可能变得可分,如男女采用不同图标,已标黑的账号以不同颜色展示,以箭头展示资金流向,边上标明数字表明资金流}
        \item{相关技能:Scala、SparkX、Pregel、JS、Django、Python}
      \end{cvitems}
    }

%---------------------------------------------------------
%---------------------------------------------------------
% \vspace{5pt}
%  \cventry
%    {} % Affiliation/role目标:将大数据技术应用于电网领域,解决传统技术无法解决的问题
%    {大数据分析技术在输变电设备状态评估中的研究及应用} % Organization/group
%     {09/2015 - 至今} % Date(s)
%    {上海,山东济南} % Location
%    {
%    \vspace{-8pt}
%      \begin{cvitems} % Description(s) of experience/contributions/knowledge
%        \item {利用局域网PC机、虚拟机搭建Hadoop分布式集群,在Hadoop集群基础上部署HBase分布式数据库}
%        \item {研究建立数据模型,存储电网复杂的数据,并尽可能支持JOIN操作与条件查询; 利用HBase二级索引技术,进行面向业务编程(TOP)}
%        \item{发表论文 A Versatile Event-Driven Data Model in HBase Database for Multi-source Data of Power Grid}
%        \item{相关技能:Java,Hadoop分布式系统,HBase分布式数据库,数据模型,二级索引(Secondary Index),面向业务编程(TOP) }
%      \end{cvitems}
%    }
%    
%    \vspace{5pt}
%    \cventry
%    {} % Affiliation/role目标:关注序列模式挖掘研究的动态,并对最前沿的算法进行并行化
%    {序列模式挖掘算法的MapReduce实现} % Organization/group
%    {09/2014 - 07/2015} % Date(s)
%    {上海} % Location
%    {
%    \vspace{-8pt}
%      \begin{cvitems} % Description(s) of experience/contributions/knowledge
%        \item {用局域网PC机、虚拟机搭建Hadoop集群;用Java语言、并行计算框架MapReduce编写序列模式挖掘算法PrefixSpan算法}
%        \item {对比MapReduce并行模式下的PrefixSpan算法与单机PrefixSpan算法的性能,研究MapReduce框架下递归算法的加速比}
%        \item{根据对PrefixSpan算法的研究,发表论文 Finding All-One Hyper-Submatrixs of An Incidence Matrix,解决0-1矩阵的全1子矩阵的问题 }
%	\item{发表论文 Finding All-One Hyper-Submatrixs of An Incidence Matrix}
%        \item{相关技术:Hadoop分布式系统, 序列模式挖掘算法, Java, 算法分析及性能分析,Latex论文撰写}
%      \end{cvitems}
%    }


%---------------------------------------------------------
\end{cventries}
