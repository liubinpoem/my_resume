%-------------------------------------------------------------------------------
%	SECTION TITLE
%-------------------------------------------------------------------------------
\vspace{-5pt}
\cvsection{项目经历}
%-------------------------------------------------------------------------------
%	CONTENT
%-------------------------------------------------------------------------------
\begin{cventries}

%---------------------------------------------------------
  \cventry
    { } % Affiliation/role目标:研究朴素贝叶斯算法对短文本分类的准确度及应用
    { 朴素贝叶斯算法对短文本自动分类的研究} % Organization/group
    {09/2015 - 07/2016} %Date(s)
    {上海} % Location
    {
    \vspace{-8pt}
      \begin{cvitems} % Description(s) of experience/contributions/knowledge
        \item {在阿里云弹性服务器上,CentOS系统中安装MySQL、Apache软件、PHP环境和Joomla建站框架}
        \item {用Python编写爬虫爬取上海交通大学BBS二手版块的帖子信息,存储到本地HBase数据库}
        \item{用C++编写朴素贝叶斯算法,利用部分帖子标题训练算法;分析朴素贝叶斯算法分类的准确度,与SVM算法的准确度进行对比}
        \item{将爬虫和算法以及训练的模型部署到服务器上,利用安卓App访问服务器,JSON格式传输数据获取分类后的结果,在Android App上展示}
        \item{相关技能:Python, Java, C++,  爬虫, HBase分布式数据库,朴素贝叶斯算法, SVM算法,LAMP环境配置,App开发}
      \end{cvitems}
    }
%---------------------------------------------------------
\vspace{5pt}
  \cventry
    {} % Affiliation/role目标:研究外卖美食畅销的主要影响因素
    {外卖美食排行榜网站开发} % Organization/group
    {09/2015 - 07/2016} % Date(s)
    {上海} % Location
    {
    \vspace{-8pt}
      \begin{cvitems} % Description(s) of experience/contributions/knowledge
        \item {利用Java编写爬虫,爬取饿了么、美团和百度外卖上在上海交通大学附近所有饭店的原始网页,存储到HBase中}
        \item {编写MapReduce程序从原始网页中提取外卖名称、销量、价格等信息,利用MapReduce的<Key,Value>键值根据销量进行排序}
        \item {将得到的结果上传到阿里云服务器,用PHP开发网页,展示外卖排行,并以超链接直接链接指向饭店的网页}
        \item {运用统计学方法以及尝试应用回归方法,分析打折减价、运费、网页位置、定价对销量的影响}
        \item {相关技能:Java, MapReduce并行计算,分布式系统,网页开发,数据分析,数学建模}
      \end{cvitems}
    }

%---------------------------------------------------------
 \vspace{5pt}
  \cventry
    {} % Affiliation/role目标:将大数据技术应用于电网领域,解决传统技术无法解决的问题
    {大数据分析技术在输变电设备状态评估中的研究及应用} % Organization/group
     {09/2015 - 至今} % Date(s)
    {上海,山东济南} % Location
    {
    \vspace{-8pt}
      \begin{cvitems} % Description(s) of experience/contributions/knowledge
        \item {利用局域网PC机、虚拟机搭建Hadoop分布式集群,在Hadoop集群基础上部署HBase分布式数据库}
        \item {研究建立数据模型,存储电网复杂的数据,并尽可能支持JOIN操作与条件查询; 利用HBase二级索引技术,进行面向业务编程(TOP)}
        \item{发表论文 A Versatile Event-Driven Data Model in HBase Database for Multi-source Data of Power Grid}
        \item{相关技能:Java,Hadoop分布式系统,HBase分布式数据库,数据模型,二级索引(Secondary Index),面向业务编程(TOP) }
      \end{cvitems}
    }
    
    \vspace{5pt}
    \cventry
    {} % Affiliation/role目标:关注序列模式挖掘研究的动态,并对最前沿的算法进行并行化
    {序列模式挖掘算法的MapReduce实现} % Organization/group
    {09/2014 - 07/2015} % Date(s)
    {上海} % Location
    {
    \vspace{-8pt}
      \begin{cvitems} % Description(s) of experience/contributions/knowledge
        \item {用局域网PC机、虚拟机搭建Hadoop集群;用Java语言、并行计算框架MapReduce编写序列模式挖掘算法PrefixSpan算法}
        \item {对比MapReduce并行模式下的PrefixSpan算法与单机PrefixSpan算法的性能,研究MapReduce框架下递归算法的加速比}
        \item{根据对PrefixSpan算法的研究,发表论文 Finding All-One Hyper-Submatrixs of An Incidence Matrix,解决0-1矩阵的全1子矩阵的问题 }
	\item{发表论文 Finding All-One Hyper-Submatrixs of An Incidence Matrix}
        \item{相关技术:Hadoop分布式系统, 序列模式挖掘算法, Java, 算法分析及性能分析,Latex论文撰写}
      \end{cvitems}
    }


%---------------------------------------------------------
\end{cventries}
